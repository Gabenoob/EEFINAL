\documentclass[12pt]{article}

% Packages
\usepackage[utf8]{inputenc}
\usepackage{xeCJK}
\usepackage{ctex}
\usepackage{amsmath, amssymb}
\usepackage{graphicx}
\usepackage{hyperref}
\usepackage{geometry}
\usepackage{float}
\usepackage{cite}

% Page layout
\geometry{a4paper, margin=1in}

% Title and Author
\title{基于BSSMF矩阵分解的图像降噪方法}
\author{王伟钊}
\date{\today}

\begin{document}

\maketitle

\begin{abstract}

\end{abstract}

\tableofcontents

\section{绪论}
图像降噪是图像处理中的一个重要问题,它在图像处理、计算机视觉、模式识别等领域有着广泛的应用。图像降噪的目的是去除图像中的噪声,使图像更加清晰,便于后续的处理。图像降噪的方法有很多,包括基于小波变换的方法、基于局部统计特性的方法、基于矩阵分解的方法等。本文提出了一种基于BSSMF矩阵分解的图像降噪方法,该方法利用了图像的低秩性和稀疏性,通过分解图像矩阵为低秩矩阵和稀疏矩阵的和,实现了对图像的降噪。

\subsection{问题描述}
在数学上,图像降噪问题可以用如下数学表达式描述\cite{SVDWhiteNoise}:
\begin{equation}
    A(i,j)=A_{0}(i,j)+N(i,j)
\end{equation}
其中$A(i,j)$是观测到的图像,$A_{0}(i,j)$是原始图像,$N(i,j)$是高斯白噪声(Addctive White Gaussian Noise, AWGN)\cite{AWGN}。

\subsection{国内外研究现状}
\subsubsection{基于传统滤波器的方法}
\subsubsection{基于小波变换的方法}
\subsubsection{基于深度学习的方法}

\section{相关工作}

\section{实验设计}

\section{实验结果与分析}

\section{结论}

\bibliographystyle{plain}
\bibliography{ref}

\appendix
\section{Appendix}
Include additional material, derivations, or data that support your paper but are not essential to the main text.


\end{document}